\subsection{IoT security}
Providing security and privacy for IoT devices is challenging mostly due to
the sheer size of the access network and supported clients (IoT devices).
Similar best practices still apply as they would for normal hosts except for
needing to work in a massively scalable and distributed network. The best
practices also take into account the computational constraints of IoT devices
to the greatest extent possible:

\begin{enumerate}
  \item	Use IEEE 802.1X for wired and wireless authentication for all devices.
  This is normally tied into a Network Access Control (NAC) architecture which
  authorizes a set of permissions per device.
  \item	Encrypt wired and wireless traffic using MACsec/IPsec as appropriate.
  \item	Maintain physical accounting of all devices, especially small ones, to
  prevent theft and reverse engineering.
  \item	Do not allow unauthorized access to sensors; ensure remote locations
  are secure also.
  \item	Provide malware protection for sensors so that the compromise of a
  single sensor is detected quickly and suppressed.
  \item	Rely on cloud-based threat analysis (again, assumes cloud is used)
  rather than a distributed model given the size of the IoT access network and
  device footprint. Sometimes this extension of the cloud is called the
  ``fog'' and encompasses other things that produce and act on IoT data.
\end{enumerate}

Another discussion point on the topic of security is determining how/where to
``connect'' an IoT network. This is going to be determined based on the
business needs, as always, but the general logic is similar to what
traditional corporate WANs use. Note that the terms ``producer-oriented'' and
``consumer-oriented'' are creations of the author and exist primarily to help
explain IoT concepts.

\begin{enumerate}
  \item	\textbf{Fully private connections:} Some IoT networks have no need
  to be accessible via the public Internet. Such examples would include
  Government sensor networks which may be deployed in a battlefield support
  capacity. More common examples might include Cisco’s ``Smart Grid''
  architecture which is used for electricity distribution and management
  within a city. Exposing such a critical resource to a highly insecure
  network offers little value since the public works department can likely
  control it from a dedicated NOC\@. System updates can be performed in-house
  and the existence of the IoT network can be (and often times, should be)
  largely unknown by the general population. In general, IoT networks that
  fall into this category are ``producer-oriented'' networks. While
  Internet-based VPNs (discussed next) could be less expensive than private
  transports, not all IoT devices can support the high computing requirements
  needed for IPsec. Additionally, some security organizations still see the
  Internet as too dirty for general transport and would prefer private,
  isolated solutions.
  \item	\textbf{Public Internet:} Other IoT networks are designed to have their
  information shared or made public between users. One example might be a
  managed thermostat service; users can log into a web portal hosted by the
  service provider to check their home heating/cooling statistics, make
  changes, pay bills, request refunds, submit service tickets, and the like.
  Other networks might be specifically targeted to sharing information
  publicly, such as fitness watches that track how long an individual
  exercises. The information could be posted publicly and linked to one’s
  social media page so others can see it. A more practical and useful example
  could include \textit{public safety information via a web portal} hosted by
  the Government. In general, IoT networks that fall into this category are
  ``consumer-oriented'' networks. Personal devices, such as fitness watches,
  are more commonly known within the general population, and they typically
  use Wi-Fi for transport.
\end{enumerate}

The topics in this section, thus far, have been on generic IoT security
considerations and solutions. Cisco identifies three core IoT security
challenges and their associated high-level solutions. These challenges are
addressed by the Cisco
\href{https://www.cisco.com/c/en/us/solutions/security/iot-threat-defense/index.html}{IoT Threat Defense},
which is designed to protect IoT environments, reducing downtime and business disruption.

\begin{enumerate}
  \item	\textbf{Antiquated equipment and technology:} Cisco recommends using
  improved visibility to help secure aging systems. Many legacy technologies use
  overly-simplistic network protocols and strategies to communicate. The
  author personally worked with electronic test equipment which used IP
  broadcasts to communicate. Because this test equipment needed to report to a
  central manager for calibration and measurement reporting, all of these
  components were placed into a common VLAN, and this VLAN was supposed to be
  dedicated only to this test equipment. Due to poor visibility (and
  convenience), other devices unrelated to the test equipment were connected
  to this VLAN and therefore forced to process the IP broadcasts. Being able
  to see this poor design and its inherent security risks is the first step
  towards fixing it. To paraphrase Cisco: Do not start with the firewall,
  start with visibility. You cannot begin segmentation until you know what is
  on your network.
  \item	\textbf{Insecure design:} Cisco recommends using access control to
  segment the network and isolate devices, systems, or networks to contain any
  security breaches between unrelated activities. For example, a small
  manufacturer may operate in a single plant where all the fabrication
  activities that feed assembly are located near one another. For industrial
  engineers skilled in production engineering but unskilled in network
  engineering, a ``flat'' network with all devices connected seems like an
  appropriate business decision. Suppose that the cutting specifications on
  one of the milling machines was malicious adjusted. Then, using the machine
  as a launch point, the attacker changed the tooling in assembly accordingly
  so that the defective part still fit snugly into the final product. The
  manufacturer is unaware of the problem until the customer receives the
  product, only to discover a defect. In summary, use ``the principle of least
  privilege'' in building communications streams between devices only as necessary.
  \item	\textbf{Lack of OT security skills:} Cisco recommends advancing the IT/OT
  convergence effort to address this issue. The previous two examples could be
  considered derivatives of this one. By intelligently and securely combining
  IT and OT within an organization, many of the relatively modern technologies
  used within IT can be leveraged within the OT environment. In addition to
  the business benefits discussed earlier, IT/OT converge can increase
  security for the OT environment at low cost. It obviates the need to deploy
  OT-specific security tools and solutions with a separate OT security team.
\end{enumerate}

The following Cisco products form the basis of the Cisco IoT Threat Defense solution set:

\textit{
\begin{enumerate}
  \item Identity Services Engine (ISE): Profiles devices and creates IoT group policies
  \item Stealthwatch: Baselines traffic and detects unusual activity
  \item Next-generation Firewall (NGFW): Identifies and blocks malicious traffic
  \item Umbrella: Analyzes DNS and web traffic
  \item Catalyst 9000 switches: Enforce segmentation and containment policies (via Trustsec)
  \item AnyConnect: Protects remote devices from threats off-net. NGFW and ISE
  team up to protect against remote threats and place remote users in the
  proper groups with proper permissions
\end{enumerate}
}
