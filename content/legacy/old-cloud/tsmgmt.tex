\subsubsection{Troubleshooting and Management}
One of the fundamental tenets of managing a cloud network is automation.
Common scripting languages, such as Python, can be used to automate a specific
management task or set of tasks. Other network device management tools, such
as Ansible, allow an administrator to create a custom script and execute it on
many devices concurrently. This is one example of the method by which
administrators can directly apply task automation in the workplace. \\

Troubleshooting a cloud network is often reliant on real-time network
analytics. Collecting network performance statistics is not a new concept, but
designing software to intelligently parse, correlate, and present the
information in a human-readable format is becoming increasingly important to
many businesses. With a good analytics engine, the NMS can move/provision
flows around the network (assuming the network is both disaggregated and
programmable) to resolve any problems. For problems that cannot be resolved
automatically, the issues are brought to the administrator’s attention using
these engines. The administrator can use other troubleshooting tools or NMS
features to isolate and repair the fault. Sometimes these analytics tools will
export reports in YAML, JSON, or XML, which can be archived for reference.
They can also be fed into in-house scripts for additional, business-specific
analysis. Put simply, analytics reduces ``data'' into ``actionable information''.
