\subsection{Security implications, compliance, and policy}
From a purely network-focused perspective, many would argue that public cloud
security is superior to private cloud security. This is the result of hiring
an organization whose entire business revolves around providing a secure,
high-performing, and highly-available network. A business where ``the network
is not the business'' may be less inclined or less interested in increasing
OPEX within the IT department, the dreaded cost center. The counter-argument
is that public cloud physical security is always questionable, even if the
digital security is strong. Should a natural disaster strike a public cloud
facility where disk drives are scattered across a large geographic region
(tornado comes to mind), what is the cloud provider’s plan to protect customer
data? What if the data is being stored in a region of the world known to have
unfriendly relations towards the home country of the supported business? These
are important questions to ask because when data is in the public cloud, the
customer never really knows exactly ``where'' the data is physically stored.
This uncertainty can be offset by using ``availability zones'' where some cloud
providers will ensure the data is confined to a given geographic region. In
many cases, this sufficiently addresses the concern for most customers, but
not always. As a customer, it is also hard to enforce and prove this. This
sometimes comes with an additional cost, too. Note that disaster recovery (DR)
is also a component of business continuity (BC) but like most things, it has
security considerations as well.

Privacy in the cloud is achieved mostly by introducing multi-tenancy
separation. Compartmentalization at the host, network, and application layers
ensure that the entire cloud architecture keeps data private; that is to say,
customers can never access data from other customers. Sometimes this
multi-tenancy can be done as crudely as separating different customers onto
different hosts, which use different VLANs and are protected behind different
virtual firewall contexts. Sometimes the security is integrated with an
application shared by many customers using some kind of public key
infrastructure (PKI). Often times maintaining this security and privacy is a
combination of many techniques. Like all things, the security posture is a
continuum which could be relaxed between tenants if, for example, the two of
them were partners and wanted to share information within the same public
cloud provider (like a cloud extranet).

The table that follows compares the security and privacy characteristics
between the different cloud deployment options.

\begin{longtable}{LLLLL}
  \toprule
  % top left square is blank
  &
  \textbf{Public Cloud}
  &
  \textbf{Private Cloud}
  &
  \textbf{Virtual Private Cloud}
  &
  \textbf{Inter-Cloud}
  \\ \midrule
  \textbf{Digital security}
  &
  Typically has best trained staff, focused on the network and not much else
  (network is the business)
  &
  Focused IT staff but likely not IT-focused upper management (network is
  likely not the business)
  &
  Coordination between clouds could provide attack surfaces, but isn’t
  wide-spread
  &
  Coordination between clouds could provide attack surfaces (like what BGPsec
  is designed to solve)
  \\ \midrule
  \textbf{Physical security}
  &
  One cannot pinpoint their data within the cloud provider’s network
  &
  Generally high as a business knows where the data is stored, breaches
  notwithstanding
  &
  Combination of public and private; depends on application component
  distribution
  &
  One cannot pinpoint their data anywhere in the world
  \\ \midrule
  \textbf{Privacy}
  &
  Transport from premises to cloud should be secured (Internet VPN, secure
  private WAN, etc.)
  &
  Generally secure assuming corporate WAN is secure
  &
  Need to ensure any replicated traffic between public/private clouds is
  protected; generally this is true with site to site VPNs
  &
  Need to ensure any replicated traffic between distributed public clouds is
  protected; customers can't perform it, but cloud providers should provide it
  \\
  \bottomrule
  \caption{Cloud Security Comparison} \\
\end{longtable}
